\documentclass[a4paper,twocolumn]{article}


\usepackage[sc]{mathpazo} % Use the Palatino font
\usepackage[T1]{fontenc} % Use 8-bit encoding that has 256 glyphs
\usepackage[utf8]{inputenc} % Use utf-8 as encoding
\linespread{1.05} % Line spacing - Palatino needs more space between lines
\usepackage{microtype} % Slightly tweak font spacing for aesthetics
\usepackage{graphicx}

\usepackage[spanish]{babel} % Language hyphenation and typographical rules
%\usepackage[galician]{babel} % Change to this if using galician

\usepackage[hmarginratio=1:1,top=32mm,columnsep=20pt]{geometry} % Document margins
\usepackage[hang, small,labelfont=bf,up,textfont=it,up]{caption} % Custom captions under/above floats in tables or figures
\usepackage{booktabs} % Horizontal rules in tables

\usepackage{enumitem} % Customized lists
\setlist[itemize]{noitemsep} % Make itemize lists more compact

\usepackage{abstract} % Allows abstract customization
\renewcommand{\abstractnamefont}{\normalfont\bfseries} % Set the "Abstract" text to bold
\renewcommand{\abstracttextfont}{\normalfont\small\itshape} % Set the abstract itself to small italic text

\usepackage{titlesec} % Allows customization of titles
\renewcommand\thesection{\Roman{section}} % Roman numerals for the sections
\renewcommand\thesubsection{\Alph{subsection}} % roman numerals for subsections
\titleformat{\section}[block]{\large\scshape\centering}{\thesection.}{1em}{} % Change the look of the section titles
\titleformat{\subsection}[block]{\large}{\thesubsection.}{1em}{} % Change the look of the section titles

\usepackage{fancyhdr} % Headers and footers
\pagestyle{fancy} % All pages have headers and footers
\fancyhead{} % Blank out the default header
\fancyfoot{} % Blank out the default footer
%\fancyhead[C]{Running title $\bullet$ May 2016 $\bullet$ Vol. XXI, No. 1} % Custom header text
\fancyfoot[C]{\thepage} % Custom footer text

\usepackage{titling} % Customizing the title section

\usepackage{hyperref} % For hyperlinks in the PDF

%----------------------------------------------------------------------------------------
%	TITLE SECTION
%----------------------------------------------------------------------------------------

\setlength{\droptitle}{-4\baselineskip} % Move the title up

\pretitle{\begin{center}\huge\bfseries} % Article title formatting
	\posttitle{\end{center}} % Article title closing formatting

\title{Desenrolle de lazos internos con optimización de operaciones de reducción} % Article title

\date{\today} % Leave empty to omit a date
\renewcommand{\maketitlehookd}{%
	\begin{abstract}
		\noindent En este informe, se abordará el tema del desenrolle de lazos para la optimización de operaciones de reducción, tal que se verá en que consiste la técnica y se comentarán experimentos.  \\\mbox{}\\
		 \textbf{\textit{Palabras clave}: bucles, desenrollamiento, optimización\ldots}
	\end{abstract}
}

%----------------------------------------------------------------------------------------

\begin{document}
	
	% Print the title
	\maketitle
	
	%----------------------------------------------------------------------------------------
	%	ARTICLE CONTENTS
	%----------------------------------------------------------------------------------------
	
	\section{Introducción}

        En el presente informe se describe una serie de experimentos diseñados para estudiar la técnica denominada desenrollamiento de lazos, en el contexto de operaciones de reducción.
        
        Para el estudio, se realizarán varios programas breves escritos en el lenguaje de bajo nivel C, ejecutados en un sistema operativo UNIX. Para la medición de los tiempos de ejecución de los bucles for se utilizará la librería \textit{sys/time.h}.

 
%	Introducción al problema tratado, incluyendo las referencias  necesarias. Por ejemplo: ``Este trabajo se basa en los estudios teóricos realizados en~\cite{Intel:2005} y \cite{spec}". En este apartado se plantean el problema a resolver, objetivos a alcanzar y metodología seguida para alcanzarlos.
	
%	Es una introducción al problema, no se desarrollarán los contenidos aquí. 
	
%	La introducción termina indicando en pocas palabras de qué secciones consta el resto del documento y de qué trata cada una. Se mencionan todas las secciones salvo la de referencias (bibliografía). 

	%------------------------------------------------
	
%-------------------------------------------

\section{Descripción de la técnica analizada}



%------------------------------------------------
	
\section{Conclusiones}



%----------------------------------------------------------------------------------------
%	Referencias
%----------------------------------------------------------------------------------------
	
\begin{thebibliography}{99} % Bibliografía - alternativamente, se recomienda el uso de bibtex o biblatex
		
	\bibitem[1]{Sistemas Operativos Modernos}
    Andrew S. Tanenbaum. 
	\newblock {\em Sistemas Operativos Modernos, 3ª edición.}
	\newblock Editorial Prentice-Hall, 2009, secciones 2.3.4 - 2.3.5.
%	\newblock \href{https://apps4two.com/curso_dba/bibliografia/2-Sistemas%20operativos%20moderno%203ed%20Tanenbaum.pdf}{https://apps4two.com/curso_dba/bibliografia/2Sistemasoperativosmoderno203ed20Tanenbaum.pdf}, 	\newblock [online]
		
\end{thebibliography}
	
	%----------------------------------------------------------------------------------------
	
\end{document}

